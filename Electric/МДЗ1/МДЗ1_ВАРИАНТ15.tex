\documentclass[a4paper, 14pt]{article}
\usepackage[utf8]{inputenc}
\usepackage{graphicx}
\usepackage[T2A]{fontenc}
\usepackage{geometry}
\usepackage{array}
\usepackage{graphicx}
\usepackage{setspace}
\usepackage{siunitx}
\usepackage{caption}
\usepackage{titlesec}
\usepackage[russian]{babel}
\usepackage{indentfirst}
\usepackage{setspace}
\usepackage{float}
\usepackage{array}
\usepackage{cellspace}
\usepackage{pgfplots}
\usepackage{pgfplotstable}
\usepackage{tocloft}
\usepackage{amsmath}
\usepackage{multicol}
\setlength{\cellspacetoplimit}{20pt}
\setlength{\cellspacebottomlimit}{20pt}

\onehalfspacing

\geometry{
  a4paper,
  left=30mm,
  right=20mm,
  top=20mm,
  bottom=20mm
}

\setlength{\parindent}{1.25cm}

\graphicspath{ {images/} }
\captionsetup[figure]{
  labelsep=period,
  justification=centering
}

\renewcommand{\figurename}{Рисунок}
\renewcommand{\contentsname}{Оглавление}

\titleformat{\chapter}[block]{\huge\bfseries\centering}{}{1em}{}
\titleformat{\section}[block]{\normalsize\bfseries}{\thesection}{1em}{}
\titleformat{\subsection}[block]{\normalsize\bfseries}{\thesubsection}{1em}{}
\titlespacing*{\section}{0pt}{\parskip}{-\parskip}
\titlespacing*{\subsection}{0pt}{\parskip}{-\parskip}
\titlespacing{\chapter}{0pt}{0pt}{0pt}

\renewcommand{\cftbeforetoctitleskip}{-30pt}
\renewcommand{\cftaftertoctitleskip}{-10pt}

\begin{document}

\begin{titlepage}
  \begin{center}
    \small
    \underline{
      \begin{minipage}{0.2\textwidth}
        \centering
        \includegraphics[width=3cm]{BMSTU logo}
      \end{minipage}
      \hfill
      \begin{minipage}{0.8\textwidth}
        \begin{center}

          \textbf{Министерство науки и высшего образования Российской Федерации
            Федеральное государственное бюджетное образовательное учреждение высшего образования «Московский государственный технический университет имени Н.Э. Баумана национальный исследовательский университет»
          (МГТУ им. Н.Э. Баумана)}
        \end{center}
      \end{minipage}

    }

    \vspace{7cm}
    \Large
    \textbf{Модульное домашнее задание №1} \\
    ПО ДИСЦИПЛИНЕ "Электротехника"\\
    \vspace{5cm}
    \normalsize
    \begin{tabular}{p{8cm}p{6cm}}
      \textbf{Студент группы ИУ1-31Б} & Соин А. Д. \\
      & «26» октября 2025 г. \\
      & \\
      \textbf{Преподаватель} & Васюков С. А. \\
      & «\underline{\hspace{1.5cm}}»\underline{\hspace{2.5cm}}2025 г. \\
    \end{tabular}
    \vfill{\Large{Москва, 2025}}
  \end{center}
\end{titlepage}

\tableofcontents
\newpage

\section{Исходные данные}
\begin{figure}[H]
  \centering
  \includegraphics{15.jpg}
\end{figure}

\section{Направления токов}
\begin{figure}[H]
  \centering
  \includegraphics{15_current.jpg}
\end{figure}

\section{Направления токов}
\begin{figure}[H]
  \centering
  \includegraphics{15_equivalent.jpg}
\end{figure}

\section{Эквивалентные сопротивления}
\subsection{Перевод исходных величин в комплексную плоскость}
\begin{itemize}
  \item $Z_{R1} = R_1 = 40 $ Ом
  \item $Z_{R6} = R_6 = 50$ Ом
  \item $Z_{R2} = R_2 = 40 $ Ом
  \item $Z_{L1} = \omega L_1 j = 50j$ Ом
  \item $Z_{L2} = \omega L_2 j = 60 j$ Ом
  \item $Z_{C3} = \frac{-j}{\omega C_2} = -20j$ Ом
  \item $Z_{C4} = \frac{-j}{\omega C_4} = -50j$ Ом
  \item $Z_{C6} = \frac{-j}{\omega C_6} = -50j$ Ом
  \item $J_1 = -4$ А
  \item $E_2 = -40 - 250j$ В
  \item $E_5 = 400 - 600j$ В
  \item $E_6 = -880 + 620j$ В
\end{itemize}
\subsection{Сопротивления в ветвях}
\begin{itemize}
  \item $Z_1 = Z_{R1} + Z_{L1} = 40 + 50j$ Ом
  \item $Z_2 = Z_{R2} + Z_{L2} = 40 + 60j$ Ом
  \item $Z_3 = Z_{C3} = - 20j$ Ом
  \item $Z_4 = Z_{C4} = - 50j$ Ом
  \item $Z_5 = 0$ Ом
  \item $Z_6 = Z_{R6} + Z_{C6} = 50 - 50j$ Ом
\end{itemize}

\section{Направления токов на Эквивалентной схеме}
\begin{figure}[H]
  \centering
  \includegraphics{15_equialent_sum_flow.png}
\end{figure}
\subsection{Токи в ветвях }
\begin{equation}
  \overline{I_{11}} = \overline{J_1} = -4
  \label{eq:green}
\end{equation}
\begin{equation}
  \overline{I_{22}}(Z_2 + Z_3 + Z_4) + \overline{I_{11}}(Z_2 + Z_3) - \overline{I_{33}}(Z_3) = \overline{E_2}
  \label{eq:red}
\end{equation}
\begin{equation}
  \overline{I_{33}}(Z_3 + Z_6) - \overline{I_{22}}Z_3 = \overline{E_5} + \overline{E_6}
  \label{eq:blue}
\end{equation}

Рассмотрим (\ref{eq:green}):
\[
  \overline{I_{22}}(40 - 10j) - 4(40 + 40j) - \overline{I_{33}}(-20j) = -40 - 250j
\]

\begin{equation}
  \overline{I_{22}}(40 -10j) - \overline{I_{33}}(-20j) = 120 - 90j
  \label{eq:red_numbers}
\end{equation}

Рассмотрим (\ref{eq:blue}):
\[
  \overline{I_{33}}(50 - 70j) - \overline{I_{22}}(-20j) - (-4)(-20j)= (400 - 600j) + (-880 + 620j)
\]
\[
  \overline{I_{33}}(50 - 70j) - \overline{I_{22}}(-20j) = -480 + 100j
\]

Отсюда $\overline{I_{22}} = 1 $ и $\overline{I_{33}} = -4 - 4j $
\subsection{Токи в ветвях}
\begin{itemize}
  \item $\overline{I_1} = \overline{I_{11}}  = -4$
  \item $\overline{I_2} = \overline{I_{11}} + \overline{I_{22}}  = -3 $
  \item $\overline{I_3} = \overline{I_{11}} + \overline{I_{22}} - \overline{I_{33}} = -7 + 4j$
  \item $\overline{I_4} = \overline{I_{22}}  = 1$
  \item $\overline{I_5} = - \overline{I_{11}} + \overline{I_{33}}  = -4j$
  \item  $\overline{I_6} = \overline{I_{33}} = -4 - 4j$
\end{itemize}
\subsection{Напряжение на источнике тока}
Найдем напряжение на источние тока по 2-ому закону Кирхгофа
\begin{equation}
  \overline{I_1}Z_1 + \overline{I_2}Z_2 + \overline{I_3}Z_3 = \overline{E_2} - \overline{E_5} + U_j
\end{equation}

\begin{equation}
  \overline{U_j} = (\overline{I_1}Z_1 + \overline{I_2}Z_2 + \overline{I_3}Z_3) + ( - \overline{E_2} + \overline{E_5}) = 240 - 750j
\end{equation}
\section{Метод узловых потенциалов}
\begin{figure}[H]
  \includegraphics{15_equialent_knot.png}
\end{figure}
Примем потенциал в точке \textbf{b} за 0 ($\phi_d = 0, \quad \phi_b = \overline{E5}$)
Рассмотрим токи в ветвях:
\begin{equation}
  \begin{cases}
    \overline{I_1} = \frac{\phi_{c}-\phi_{d}}{Z_1 + \infty} + \overline{J_1} \\
    \overline{I_2} = \frac{\phi_{a}-\phi_{c} + E_2}{Z_2 + 0} \\
    \overline{I_3} = \frac{\phi_{a}-\phi_{b}}{Z_3} \\
    \overline{I_4} = \frac{\phi_{c}-\phi_{b}}{Z_4} \\
    \overline{I_6} = \frac{\phi_{d}-\phi_{a} + E_6}{Z_6 + 0} \\
  \end{cases}
\end{equation}

Рассмотрим узел \textbf{c}:
\begin{equation*}
  \overline{I_1} + \overline{I_4} - \overline{I_2} = 0
\end{equation*}
\begin{equation*}
  \frac{\phi_{c}-\phi_{d}}{Z_1 + \infty} + \overline{J_1} + \frac{\phi_{c}-\phi_{b}}{Z_4} - \frac{\phi_{a}-\phi_{c} + E_2}{Z_2 + 0} = 0
\end{equation*}

\begin{equation}
  \phi_{a} \frac{1}{Z_2 + 0} + \phi_{b} \frac{1}{Z_4} - \phi_{c}(\frac{1}{Z_1 + \infty} + \frac{1}{Z_4} + \frac{1}{Z_2 + 0})  + \phi_{d} \frac{1}{Z_1 + \infty} = \overline{J_1} + \frac{E_2}{Z_2 + 0}
\end{equation}

Рассмотрим узел \textbf{a}:
\begin{equation*}
  \overline{I_2} - \overline{I_3} - \overline{I_6} = 0
\end{equation*}
\begin{equation*}
  \frac{\phi_{a}-\phi_{c} + E_2}{Z_2 + 0} - \frac{\phi_{a}-\phi_{b}}{Z_3} - \frac{\phi_{d}-\phi_{a} + E_6}{Z_6 + 0} = 0
\end{equation*}

\begin{equation}
  \phi_{a} (\frac{1}{Z_3} + \frac{1}{Z_6-0} - \frac{1}{Z_2 + 0}) + \phi_{b} \frac{1}{Z_3} + \phi_{c} \frac{1}{Z_2 + 0}   + \phi_{d} \frac{1}{Z_6 + 0} = - \frac{E_2}{Z_2 + 0} + \frac{E_6}{Z_6 + 0}
\end{equation}

Получаем следующие уравнения:
\begin{equation}
  \begin{cases}
    \phi_{a} \frac{1}{Z_2 + 0} + \phi_{b} \frac{1}{Z_4} - \phi_{c}(\frac{1}{Z_1 + \infty} + \frac{1}{Z_4} + \frac{1}{Z_2 + 0})  + \phi_{d} \frac{1}{Z_1 + \infty} = \overline{J_1} + \frac{E_2}{Z_2 + 0} \\
    \phi_{a} (\frac{1}{Z_3} + \frac{1}{Z_6-0} - \frac{1}{Z_2 + 0}) + \phi_{b} \frac{1}{Z_3} + \phi_{c} \frac{1}{Z_2 + 0}   + \phi_{d} \frac{1}{Z_6 + 0} = - \frac{E_2}{Z_2 + 0} + \frac{E_6}{Z_6 + 0} \\
    \phi_d = 0\\
    \phi_b = \overline{E5}
  \end{cases}
\end{equation}

\section{Метод эквивалентного генератора}

\begin{figure}[H]
  \centering
  \includegraphics{15_equialent_gen.png}
\end{figure}

Контурные токи
\begin{equation}
  \begin{cases}
    \overline{I'_{22}} = \overline{J_{1}} = -4 \\
    \overline{I'_{11}} = \frac{\overline{E_6}  + \overline{E_5} - \overline{I'_{22}} Z_6}{Z_6 + Z3} = \frac{ (400 - 600j) + (-880 + 620j) - (-4) (50-50j)}{(50 - 50j) + (-20j)} = \frac{-280 - 180j}{50 - 70j}
  \end{cases}
\end{equation}

Отсюда

\begin{equation}
  \begin{cases}
    \overline{I_{2}} = \overline{I'_{22}} = -4 \\
    \overline{I_{3}} = \overline{I'_{11}} = - \frac{-280 - 180j}{50 - 70j} =  \frac{+280 + 180j}{50 - 70j} \\

  \end{cases}
\end{equation}

Рссмотрим контур включающий ветви 2, 3 и разрыв $U_{bc}$, напрвление обхода по часовой стрелке
\begin{equation}
  U_{bc} + I_2 Z_2 + I_3 Z_3 = E_2
\end{equation}

\begin{equation}
  U_{bc} = E_2 - I_2 Z_2 - I_3 Z_3 = (-40 - 250j) - (-4)(40 + 60j) - \frac{+280 +180j}{50 - 70j} (-20j) = \frac{1580 - 230i}{37}
\end{equation}

Эквивалентное сопротивление
\begin{equation}
  Z_{экв} = Z_2 + \frac{Z_3Z_6}{Z_3+Z_6} = (40+60j) + \frac{(-20j)(50-50j)}{(-20j)+ (50-50j)} = \frac{1580 + 1620j}{37}
\end{equation}

Найдем ток в вевти 4
\begin{equation}
  \overline{I_4} = \frac{U_{bc}}{Z_{экв}  + Z_4} = \frac{\frac{1580 - 230i}{37}}{\frac{1580 + 1620i}{37} + (-50j)} = 1 А
\end{equation}

\section{Проверка}

В результате решения получены следующие токи:
\begin{multicols}{2}

  \raggedcolumns
  \begin{itemize}
    \item $\overline{I_1} = -4$
    \item $\overline{I_2} = -3 $
    \item $\overline{I_3} = -7 + 4j$
    \item $\overline{I_4} = 1$
    \item $\overline{I_5} = -4j$
    \item $\overline{I_6} = -4 - 4j$
  \end{itemize}

  \begin{figure}[H]
    \begin{center}
      \includegraphics[width=0.5\textwidth]{check.png}
    \end{center}
  \end{figure}

\end{multicols}
\end{document}