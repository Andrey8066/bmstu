\documentclass[a4paper, ]{extreport}
\usepackage[utf8]{inputenc}
\usepackage{graphicx}
\usepackage[T2A]{fontenc}
\usepackage{geometry}
\usepackage{array}
\usepackage{graphicx}
\usepackage{setspace}
\usepackage{siunitx}
\usepackage{caption}
\usepackage{titlesec}
\usepackage[russian]{babel}
\usepackage{indentfirst}
\usepackage{setspace}
\usepackage{float}
\usepackage{array}
\usepackage{cellspace}
\usepackage{pgfplots}
\usepackage{pgfplotstable}
\usepackage{tocloft}
\usepackage{amsmath}
\usepackage{multicol}
\usepackage{minted}
\setlength{\cellspacetoplimit}{20pt}
\setlength{\cellspacebottomlimit}{20pt}

\onehalfspacing

\geometry{
  a4paper,
  left=30mm,
  right=20mm,
  top=20mm,
  bottom=20mm
}

\setlength{\parindent}{1.25cm}

\graphicspath{ {images/} }
\captionsetup[figure]{
  labelsep=period,
  justification=centering
}

\renewcommand{\figurename}{Рисунок}
\renewcommand{\contentsname}{Оглавление}

\titleformat{\chapter}[block]{\huge\bfseries\centering}{}{1em}{}
\titleformat{\section}[block]{\normalsize\bfseries}{\thesection}{1em}{}
\titleformat{\subsection}[block]{\normalsize\bfseries}{\thesubsection}{1em}{}
\titlespacing*{\section}{0pt}{\parskip}{-\parskip}
\titlespacing*{\subsection}{0pt}{\parskip}{-\parskip}
\titlespacing{\chapter}{0pt}{0pt}{0pt}

\renewcommand{\cftbeforetoctitleskip}{-30pt}
\renewcommand{\cftaftertoctitleskip}{-10pt}

\begin{document}

\begin{titlepage}
  \begin{center}
    \small
    \underline{
      \begin{minipage}{0.2\textwidth}
        \centering
        \includegraphics[width=3cm]{BMSTU logo}
      \end{minipage}
      \hfill
      \begin{minipage}{0.8\textwidth}
        \begin{center}

          \textbf{Министерство науки и высшего образования Российской Федерации
            Федеральное государственное бюджетное образовательное учреждение высшего образования «Московский государственный технический университет имени Н.Э. Баумана национальный исследовательский университет»
          (МГТУ им. Н.Э. Баумана)}
        \end{center}
      \end{minipage}

    }

    \vspace{7cm}
    \Large
    \textbf{ОТЧЕТ ПО УЧЕБНОЙ ПРАКТИКЕ}\\
    \vspace{5cm}
    \normalsize
    \begin{tabular}{p{8cm}p{6cm}}
      \textbf{Студент группы ИУ1-31Б} & Соин А. Д. \\
      & «14» декабря 2025 г. \\
      & \\
      \textbf{Преподаватель} & Замараев \\
      & «\underline{\hspace{1.5cm}}»\underline{\hspace{2.5cm}}2025 г. \\
    \end{tabular}
    \vfill{\Large{Москва, 2025}}
  \end{center}
\end{titlepage}

\tableofcontents
\newpage

\chapter{Теория}
\section{Функции Matlab}
\subsection{}

\section{Функции Simulink}

\subsection{Scope (Осциллограф)}

Блок Scope предназначен для визуализации сигналов во времени в процессе моделирования.

Основные функции

отображение одного или нескольких сигналов;

анализ переходных процессов;

проверка устойчивости и качества регулирования.

\textbf{Особенности}

\begin{itemize}
  \item поддерживает несколько входных каналов;
  \item масштабирование по времени и амплитуде;
  \item возможность экспорта данных в MATLAB Workspace.
\end{itemize}
Типичное применение

Анализ реакции системы на уставку, возмущение, переходный процесс.

\subsection{Constant (Константа)}

Блок Constant формирует постоянный сигнал, не зависящий от времени.

Параметры

значение константы (скаляр, вектор или матрица);

тип данных (double, int и др.).

Типичное применение

задание уставки;

постоянные параметры системы;

начальные условия.

\subsection{Sum (Сумматор)}

Блок Sum выполняет алгебраическое сложение и вычитание входных сигналов.

\textbf{Особенности}

\begin{itemize}
  \item настраиваемое количество входов;
  \item задание знаков входов (+, -);
  \item поддержка векторных сигналов.
\end{itemize}

\subsection{Integrator (Интегратор)}

Блок Integrator реализует непрерывное интегрирование входного сигнала:

Параметры

начальное значение;

ограничения (насыщение);

режим сброса (reset).

Типичное применение

моделирование динамики (скорость → координата);

реализация I-звена регулятора;

моделирование физических процессов.

\subsection{Demux (Демультиплексор)}

Блок Demux разделяет векторный сигнал на несколько скалярных выходов.

\textbf{Особенности}

\begin{itemize}
  \item число выходов задаётся пользователем;
  \item сохраняет порядок элементов вектора.
\end{itemize}

\subsection{Subsystem (Подсистема)}

Блок Subsystem служит для иерархической организации модели.

\textbf{Типы}
\begin{itemize}
  \item обычная Subsystem;
  \item Enabled / Triggered Subsystem;
  \item While / For Iterator Subsystem.
\end{itemize}

\subsection{MATLAB Function (Пользовательская функция)}

Блок MATLAB Function позволяет реализовать алгоритмы на языке MATLAB внутри модели Simulink.

\textbf{Особенности}
\begin{itemize}
  \item поддержка условий (if, switch, while);
  \item работа с векторами и матрицами;
  \item автоматическая проверка размерностей;
\end{itemize}
поддержка генерации кода (Embedded Coder).

\end{document}